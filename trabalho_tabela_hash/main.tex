\documentclass{article}
\usepackage{graphicx} % Required for inserting images

\title{Relatório Tabela Hash}
\author{Guilherme Ferraz Freire De Araújo}
\date{October 2024}

\begin{document}

\maketitle

\section{Introdução}
Este é um trabalho feito em Java sobre a Tabela Hash, onde deveríamos implementar a tabela e analisar seu desempenho em diferentes casos. O meu código ficou separado em 4 arquivos: Registro, No, TabelaHash e o Main.

\section{Registro}
A classe Registro irá gerar uma chave para a tabela, que terá o nome de código.

\section{No}
A classe No irá representar uma lista encadeada, e nela há 2 atributos: registro e próximo.

O registro será responsável por armazenar os dados do registro.

O próximo faz referência ao próximo No presente na lista, e definimos aqui que o valor do próximo será null.

\section{Tabela Hash}
Aqui será onde eu defini minha tabela hash.

Aqui teremos 3 funções de hash (hash1, hash2, hash3), cada uma das funções terá uma chave em um índice dentro do tamanho da tabela.
\begin{itemize}
    \item O hash1 usará o operador módulo.
    \item O hash2 será usado para multiplicar.
    \item O hash3 será usado para realizar operações com bit.
\end{itemize}

O método Inserir será onde iremos inserir um novo registro à tabela, ele vai usar a função de hash para determinar onde o registro deverá estar. Caso um elemento esteja em uma determinada posição e for colocado um novo elemento na mesma, o novo elemento será mandado para o final da lista.

O Buscar irá buscar um registro com base no seu código, aqui o hash será usado para determinar onde o registro deverá estar, caso seja necessário, ela percorre a lista ligada nessa posição para encontrar o registro desejado.

O método contagem de colisões servirá para contar quantas vezes ocorrerão colisões na tabela, as colisões acontecem quando dois ou mais elementos são mapeados para a mesma posição na tabela.

O método calcularHash seleciona qual das três funções de hash será usada, com base na configuração inicial da tabela.

\section{Main}
\subsection{Configuração Inicial}
O código define algumas constantes no início:
\begin{itemize}
    \item TAMANHOS: diferentes tamanhos para a tabela hash (10.000, 100.000, 1.000.000).
    \item QUANTIDADES: diferentes quantidades de dados a serem inseridos (1 milhão, 5 milhões, 20 milhões).
    \item NUM\_FUNCOES\_HASH: número de funções de hash a serem testadas (3).
    \item NUM\_BUSCAS: número de buscas a serem realizadas em cada teste (5).
\end{itemize}

Aqui teremos um gerador aleatório de números com uma semente fixa (42) para garantir que os mesmos números sejam gerados em cada execução.

\subsection{Execução}
Ao longo da execução o programa executa vários loops aninhados para testar diferentes combinações:
\begin{itemize}
    \item Para cada tamanho de tabela
    \item Para cada quantidade de dados
    \item Para cada função de hash
\end{itemize}

Para cada combinação, o programa vai gerar um conjunto de dados aleatórios, criar uma nova tabela hash, medir o tempo de inserção de todos os dados, contar o número de colisões e realizar buscas aleatórias e medir o tempo médio de busca.

Eu usei o System.nanoTime() para medir o tempo de execução em nanossegundos.

\subsection{Resultados}
Agora que tudo foi explicado, os resultados serão exibidos nessa ordem:
\begin{itemize}
    \item O tamanho da tabela, quantidade de dados e função de hash usada.
    \item O tempo total de inserção e número de colisões.
    \item O tempo médio de busca.
\end{itemize}

\end{document}